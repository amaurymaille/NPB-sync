\documentclass[a4paper,11pt]{article}

\usepackage[english]{babel}
\usepackage[T1]{fontenc}
\usepackage[utf8]{inputenc}
\usepackage{fullpage}
\usepackage{eufrak}
\usepackage{amssymb}
\usepackage{amsfonts}
\usepackage{amsthm}
\usepackage{amsmath}
\usepackage{listings}
\usepackage{xcolor}

\lstset{
  basicstyle=\tt\footnotesize,
  frame=single
}

\begin{document}

\pagenumbering{arabic}

\paragraph{Code structure} MWE.

\begin{lstlisting}[language=c,numbers=left,tabsize=2]
void producer(NaiveQueueImpl<int64_t>* queue, Observer<int64_t>* observer) {
  for (uint64_t i = 0; i < I; ++i) {
    int64_t value = 0;
    for (volatile int j = 0; j < work_limit(); ++j)
      ;
    queue->push(value);
  }

  queue->terminate();
}

void consumer(NaiveQueueImpl<int64_t>* queue) {
  while (true) {
    std::optional<int64_t> value = queue->pop();
    if (!value)
      break;

    for (volatile int j = 0; j < work_limit(); ++j)
      ;
  }
}
\end{lstlisting}

\paragraph{Notations}

\begin{itemize}
    \item $ I_p $ is the total number of iterations of a producer (\texttt{work\_limit}).
    \item $ I_c $ is the total number of iterations of a consumer (\texttt{work\_limit}).
    \item $ CostP_x $ is the time required to perform a local push / a local pull for entity x. \textcolor{red}{We assume that this value is the same for local pulls, local pushs, and for all producers and consumers}.
    \item $ W_{p_i} $ is the amount of work performed by producer thread $ i $. \textcolor{red}{We assume this value is the same for all producers}.
    \item $ W_{c_i} $ is the amount of work performed by consumer thread $ i $. \textcolor{red}{We assume this value is the same for all consumers}.
    \item $ S_p $ is the step of all producers.
    \item $ S_c $ is the step of all consumers.
    \item $ N_p $ is the amount of producers
    \item $ N_c $ is the amount of consumers
    \item $ S $ is the total amount of data that is transfered over a FIFO. $ S = N_p * S_p = N_c * S_c $.
    \item $ N $ is the total number of threads ($ N = N_p + N_c $)
    \item $ CostS $ is the time taken to perform a synchronization, minus the time spent in the critical section itself. This includes the wait time to lock, the time to lock and the time to unlock the mutex.
    \item $ CostCC $ is the time spent in the critical section itself. This is basically the time to transfer from/to the local FIFO.
\end{itemize}

\paragraph{Execution time}
Execution time depends on the amount of work of the consumers / producers and the amount of producers and consumers. First, compute : 
\medskip

    $ CPRatio = N_p * \overline{W_{c_i}} $ and $ PCRatio = N_c * \overline{W_{p_i}} $.

    \medskip

    Our formulas are valid under the following assumptions / runtime rules : 

    \begin{center}
        \fbox{
            \parbox{0.4\textwidth}{
                \begin{align*}
                    N_c * S_c &= N_p * S_p\\
                    \forall x, CostCC_x &\equiv 2 * CostP_x * S_x\\
                    \forall i, j, CostP_i &= CostP_j
                \end{align*}
            }
        }
    \end{center}

        \medskip
\noindent \textbf{If} $ PCRatio > CPRatio $ \textbf{then}

    % $ time(S) = S_p * (\overline{W_{p_i}} + CostP) + N_p * (CostCC_{S_p} + CostS) + (N_c - 1) * (CostCC_{S_c} + CostS) + \frac{I_c}{S_c} * (CostCC_{S_c} + CostS + S_c * (\overline{W_{c_i}} + CostP)) $ 
    $ time(S_p) = I_p * (\overline{W_{p_i}} + CostP) + \frac{I_p}{S_p} * (CostCC_p + CostS) + S_c * (\overline{W_{c_i}} + CostP) + (N_p - 1) * (CostS + CostCC_p * S_p) + N_c * (CostS + CostCC_c) $

    \begin{itemize}
        \item $ I_p * (\overline{W_{i_p}} + CostP) $ is the amount of time required for a producer to produce all its items and insert them inside its local FIFO. Note that this uses $ \overline{W_{p_i}} $.
        \item $ \frac{I_p}{S_p} * (CostCC_p + CostS)) $ is the amount of time required to transfer all the elements from the local buffer into the shared buffer, which includes the synchronization time $ CostS $. This synchronization is performed $ \frac{I_p}{S_p} $ times since the step is used to control the amount of synchronization.
        \item $ (N_p - 1) * (CostS + CostCC_p * S_p) $ is the amount of time required for all other producers to complete their critical sections.
        \item $ N_c * (CostS + CostCC_c) $ is the amount of time required for all consumers to complete their critical sections.
        \item $ S_c * (\overline{W_{c_i}} + CostP) $ is the amount of time required by the last consumer to complete its work. Note that his uses $ \overline{W_{c_i}} $.
    \end{itemize}

    $ time'(S_p) = - \frac{I_p * CostS}{{S_p}^{2}} + \frac{N_p * \overline{W_{c_i}} + N_p * CostP}{N_c} + 2 * (N_p * CostP - CostP + N_p * CostP) \Rightarrow $\\
    $ time'(S_p) = 0 \equiv S_p = \sqrt{\left | \frac{I_p * Cost_S}{\frac{N_p * W_c + N_p * CostP}{N_C} + 2 * (N_p * CostP - CostP + N_p * CostP)} \right | } $ 

    \medskip
\noindent \textbf{Else}

    % $ time(S) = S_c * (\overline{W_{c_i}} + CostP) + N_c * (CostCC_{S_c} + CostS) + (N_p - 1) * (CostCC_{S_p} + CostS) + \frac{I_p}{S_p} * (CostCC_{S_p} + CostS + S_p * (\overline{W_{p_i}} + CostP)) $ 

    % $ time(S) = W_i * I + CostP * I + \frac{I}{S} * (CostCC + CostS) + S * (W_i + CostP) + (N - 1) * (CostS + CostP * S) $
    
\medskip

\paragraph{Validity of previous formula}
The above formula is valid under the assumption that $ W_i $ respects the following inequation : 

    $ valid(S) = W_i \geq \frac{(N - 1) * (CostCC + CostS)}{S} - CostP $

\paragraph{CostCC estimation}

CostCC may be estimated based on the value of $ S $. In general :

$ CostCC(S) = \alpha * \frac{S}{ln(S) + \beta} $

First approximation : $ \alpha \approx 6.5 $.

\subparagraph{Alpha estimation}

\begin{enumerate}
    \item Fixate a value for $ \beta $
    \item Select a step $ S $
    \item $ \alpha = \frac{CostCC(S) * ln(S) + \beta}{S} $
\end{enumerate}

\paragraph{Finding the optimal step}

$ time'(S) = W_i + N * CostP - \frac{I * CostS}{S^2} $

$ time'(S) = 0 \equiv S = \sqrt{\frac{I * CostS}{N * CostP + Wi}} $

\end{document}
